\documentclass{classrep}
\usepackage[utf8]{inputenc}
\usepackage{color}
\usepackage[polish]{babel}
\usepackage[T1]{fontenc}
\usepackage{enumerate}
\usepackage{graphicx}
\usepackage{float}
\restylefloat{table}
\usepackage{multirow}
\usepackage[normalem]{ulem}
\useunder{\uline}{\ul}{}
\selectlanguage{polish}

\studycycle{Informatyka, studia dzienne, I st.}
\coursesemester{VI}

\coursename{Komputerowe systemy rozpoznawania}
\courseyear{2018/2019}

\courseteacher{dr hab. inż. Adam Niewiadomski}
\coursegroup{poniedziałek, 12:15}

\author{
  \studentinfo{Stanisław Zakrzewski}{210360} \and
  \studentinfo{ Maciej Socha}{210321}
}

\title{Zadanie 1: Ekstrakcja cech, miary podobieństwa, klasyfikacja}

\begin{document}
\maketitle

\section{Cel}
Celem zadania było poznanie oraz zaimplementowanie różnych metod ekstrakcji cech z tekstów, określania podobieństwa oraz klasyfikacji tekstów.

\section{Wprowadzenie}
{\color{blue}
We wprowadzeniu należy zaprezentować całą teorię potrzebną do realizacji
zadania (przy czym należy tu ograniczyć się wyłącznie do tego, co było
wykorzystane) tak aby osoba, która nigdy wcześniej nie zetknęła się z tą
tematyką, potrafiła zrozumieć dalszy opis. Część ta powinna wprowadzać
wszystkie wykorzystywane wzory, oznaczenia itp., do których należy się
odwoływać w dalszej części niniejszgo sprawozdania. Zamieszczony tu własny
opis teorii (a nie skopiowany!) należy poprzeć odwołaniami bibliograficznymi
do literatury zamieszczonej na końcu. }

\section{Opis implementacji}
Implementacja została stworzona w języku Java. Poniżej został umieszczony diagram klas. Są na nim tylko istotne dla danego zagadnienia klasy.

\section{Materiały i metody}
{\color{blue}
W tym miejscu należy opisać, jak przeprowadzone zostały wszystkie badania,
których wyniki i dyskusja zamieszczane są w dalszych sekcjach. Opis ten
powinien być na tyle dokładny, aby osoba czytająca go potrafiła wszystkie
przeprowadzone badania samodzielnie powtórzyć w celu zweryfikowania ich
poprawności (a zatem m.in. należy zamieścić tu opis architektury sieci,
wartości współczynników użytych w kolejnych eksperymentach, sposób
inicjalizacji wag, metodę uczenia itp. oraz informacje o danych, na których
prowadzone były badania). Przy opisie należy odwoływać się i stosować do
opisanych w sekcji drugiej wzorów i oznaczeń, a także w jasny sposób opisać
cel konkretnego testu. Najlepiej byłoby wyraźnie wyszczególnić (ponumerować)
poszczególne eksperymenty tak, aby łatwo było się do nich odwoływać dalej.}

\section{Wyniki}
{\color{blue}
W tej sekcji należy zaprezentować, dla każdego przeprowadzonego eksperymentu,
kompletny zestaw wyników w postaci tabel, wykresów itp. Powinny być one tak
ponazywane, aby było wiadomo, do czego się odnoszą. Wszystkie tabele i wykresy
należy oczywiście opisać (opisać co jest na osiach, w kolumnach itd.) stosując
się do przyjętych wcześniej oznaczeń. Nie należy tu komentować i interpretować
wyników, gdyż miejsce na to jest w kolejnej sekcji. Tu również dobrze jest
wprowadzić oznaczenia (tabel, wykresów) aby móc się do nich odwoływać
poniżej.}

\section{Dyskusja}
{\color{blue}
Sekcja ta powinna zawierać dokładną interpretację uzyskanych wyników
eksperymentów wraz ze szczegółowymi wnioskami z nich płynącymi. Najcenniejsze
są, rzecz jasna, wnioski o charakterze uniwersalnym, które mogą być istotne
przy innych, podobnych zadaniach. Należy również omówić i wyjaśnić wszystkie
napotakane problemy (jeśli takie były). Każdy wniosek powinien mieć poparcie
we wcześniej przeprowadzonych eksperymentach (odwołania do konkretnych
wyników). Jest to jedna z najważniejszych sekcji tego sprawozdania, gdyż
prezentuje poziom zrozumienia badanego problemu.}

\section{Wnioski}
{\color{blue}W tej, przedostatniej, sekcji należy zamieścić podsumowanie
najważniejszych wniosków z sekcji poprzedniej. Najlepiej jest je po prostu
wypunktować. Znów, tak jak poprzednio, najistotniejsze są wnioski o
charakterze uniwersalnym.}


\begin{thebibliography}{0}

  \bibitem{article1} David D. Lewis
  	\textsl{Feature Selection and Feature Extract ion for Text Categorization}, University of Chicago\
  	\url {https://aclweb.org/anthology/H92-1041?fbclid=IwAR248ftiyFqXrFpi51IDLorT7Ngso369BPTOaOeSYE3QGG1gYD9TNfy58qc}
  	
  \bibitem{article2} David Dolan Lewis
  	\textsl{Representation and learning in information retrieval}, University of Massachusetts\
  	\url {http://ciir.cs.umass.edu/pubfiles/UM-CS-1991-093.pdf}
  	
  \bibitem{article3} David D. Lewis
  	\textsl{Data Extraction as Text Categorization : An
Experiment With the MUC-3 Corpus}, University of Chicago\
	\url {https://www.aclweb.org/anthology/M91-1035}
	
  \bibitem{article4} Marina Sokolova, Guy Lapalme
  	\textsl{A systematic analysis of performance measures for classification tasks}, Information Processing and Management no 45\
  	\url {http://rali.iro.umontreal.ca/rali/sites/default/files/publis/SokolovaLapalme-JIPM09.pdf?fbclid=IwAR2M7_a4QxL_F4yCOB_Akp4ghkoUKrBnHT9xzCfuTcoVrLBe3lN3kIlPt00}
  	
  \bibitem{stanford} \url {https://stanfordnlp.github.io/CoreNLP}
  

\end{thebibliography}
{\color{blue} 
\end{document}
